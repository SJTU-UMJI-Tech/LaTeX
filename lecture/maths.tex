\section{Use Maths in \LaTeX}
\begin{frame}
	\tableofcontents[currentsection,hideothersubsections]
\end{frame}

\subsection{Equation}

\begin{frame}
	\frametitle{The equation environment}
	An \structure{equation} environment contains a set of maths equations
	\begin{command}
		\samplecommand{begin}\{equation(\structure{*})\}\\
		\qquad ...\\
		\samplecommand{end}\{equation(\structure{*})\}\\
	\end{command}
	\begin{example}
		\begin{equation}
		curl\ F=\left(\frac{\partial F_z}{\partial y}-\frac{\partial F_y}{\partial z}\right)\hat{n_x}+\left(\frac{\partial F_x}{\partial z}-\frac{\partial F_z}{\partial x}\right)\hat{n_y}+\left(\frac{\partial F_y}{\partial x}-\frac{\partial F_x}{\partial y}\right)\hat{n_z}
		\end{equation}
	\end{example}
	If a star(\structure{*}) is added, the sequence number of the equation won't be displayed. Note that the environment name in the \samplecommand{begin} and \samplecommand{end} statements must be the same(both or neither have a \structure{*} here).
\end{frame}

\begin{frame}
	The \LaTeX\ script of the equation above is quite long, but not so difficult as you think so, while how I display the script to you is far more confusing, and you may check it in the tex file of the lecture sildes
	\begin{align*}
	\color{blue} curl\backslash\ F
	\text{=}\backslash left & \color{blue} (\backslash frac\{\backslash partial\ F\uline{\ }z\}\{\backslash partial\ y\}\\
	& \color{blue} \text{-}\backslash frac\{\backslash partial\ F\_y\}\{\backslash partial\ z\}\backslash right)\backslash hat\{n\_x\}\\
	\color{blue} \text{+}\backslash left & \color{blue} (\backslash frac\{\backslash partial\ F\_x\}\{\backslash partial\ z\}\\
	& \color{blue} \text{-}\backslash frac\{\backslash partial\ F\_z\}\{\backslash partial\ x\}\backslash right)\backslash hat\{n\_y\}\\
	\color{blue} \text{+}\backslash left & \color{blue} (\backslash frac\{\backslash partial\ F\_y\}\{\backslash partial\ x\}\\
	& \color{blue} \text{-}\backslash frac\{\backslash partial\ F\_x\}\{\backslash partial\ y\}\backslash right)\backslash hat\{n\_z\}
	\end{align*}
	In the script, only a space after \structure{\textbackslash} will be printed as a space, \structure{\textbackslash partial} prints the symbol \structure{$\partial$}, \structure{\textbackslash frac\{...\}\{...\}} makes a \structure{fraction}, \structure{\textbackslash left(} and \structure{\textbackslash right)} makes \structure{brackets} (of course they can be nested and must be in couple, but you can use two kinds of brackets on the both side, i.e., \structure{\textbackslash left[} and \structure{\textbackslash right\textbackslash rbrace}, in which you must use \structure{\textbackslash rbrace} or \structure{\textbackslash \}} to print a right brace \structure{$\rbrace$} \\
\end{frame}

\begin{frame}
	How about equations with multiple lines?\\
	The \structure{aligned} environment can be used.
	\begin{example}
		\begin{equation}
		\left\lbrace\begin{aligned}
			x+y&=1\\x-y&=1
		\end{aligned}\right.\Longrightarrow
		\left\lbrace\begin{aligned}
			x&=1\\y&=0
		\end{aligned}\right.
		\end{equation}
	\end{example}
	\begin{align*}
		&\color{blue} \backslash left \backslash lbrace \backslash begin\{aligned\}\\
		&\color{blue} \quad\quad\text{x+y\&=1} \backslash \backslash \text{x-y\&=1}\\
		&\color{blue} \backslash end\{aligned\}\backslash right.\backslash Longrightarrow\\
		&\color{blue} \backslash left \backslash lbrace \backslash begin\{aligned\}\\
		&\color{blue} \quad\quad\text{x\&=1} \backslash \backslash \text{y\&=0}\\
		&\color{blue} \backslash end\{aligned\}\backslash right.
	\end{align*}
	We can use a dot(\structure{.}) when we want to insert nothing in one of the brackets.
\end{frame}

\begin{frame}
	\frametitle{Something more about equation environment}
    What if the space between equation and the main body paragraph is considered larger than expectation? Is there any way to modify the line spacing?
\\In default style of equation is like
    \begin{example}
    your body paragraph is supposed to be typed here
        \begin{equation}
         a \times b =c
        \end{equation}
    your body paragraph is supposed to be typed here
	\end{example}
\end{frame}

\begin{frame}
	But if we add \\
	\samplecommand{setlength}\samplecommand{abovedisplayskip}\{\structure{pt}\} or\\
	\samplecommand{setlength}\samplecommand{belowdisplayskip}\{\structure{pt}\}, we have
    \begin{example}
		\ your body paragraph is supposed to be typed here
		{\setlength\abovedisplayskip{0pt}
        \setlength\belowdisplayskip{0pt}
        \begin{equation}
         a \times b =c
        \end{equation}}
		your body paragraph is supposed to be typed here
	\end{example}
    {\color{blue}\{$\backslash$setlength$\backslash$abovedisplayskip\{0pt\}
        \\$\backslash$setlength$\backslash$belowdisplayskip\{0pt\}
       \\ $\backslash$begin\{equation\}
         \\a $\backslash$times b =c
        \\$\backslash$end\{equation\}\}}
    \\[0.5em]The margin between the body paragraphs and the equation will be lessened 
    as is in the example.
\end{frame}

\subsection{Align}

\begin{frame}
	\frametitle{The align/aligned environment}
	An \structure{align} environment is used outside a maths environment like \structure{equation}
	\begin{command}
		\samplecommand{begin}\{align(\structure{*})\}\\
		\qquad ...\\
		\samplecommand{end}\{align(\structure{*})\}\\
	\end{command}
	An \structure{aligned} environment is used inside a maths environment like \structure{equation}, it is known as an \structure{inline} environment.
	\begin{command}
		\samplecommand{begin}\{equation(\structure{*})\}\\
		\qquad\samplecommand{begin}\{aligned\}\\
		\qquad\qquad ...\\
		\qquad\samplecommand{end}\{aligned\}\\
		\samplecommand{end}\{equation(\structure{*})\}\\
	\end{command}
\end{frame}

\begin{frame}
	The \structure{align/aligned} environment is a basic align and multiline environment.\\
	\begin{example}
		\begin{flalign}
			a+b&\Leftrightarrow b+a\\
			(a+b)+c&\Leftrightarrow a+(b+c)
		\end{flalign}
	\end{example}
	\begin{align*}
		&\color{blue} \backslash begin\{align\}\\
		&\color{blue} \quad\quad a\text{+}b\ \&\ \backslash Leftrightarrow\ b\text{+}a\ \backslash\backslash \\
		&\color{blue} \quad\quad (a\text{+}b)\text{+}c\ \&\ \backslash Leftrightarrow\ a\text{+}(b\text{+}c)\\
		&\color{blue} \backslash end\{align\}
	\end{align*}
	In order to make a new line, you can easily use \structure{\textbackslash\textbackslash} where you'd like (but not in certain maths environments such as \structure{equation}). \structure{\&} is used to align the equations, you can use multiple \structure{\&}s and the \structure{\&}s on every line will be aligned respectively.
\end{frame}

\subsection{Inline}

\begin{frame}
	\frametitle{A simple method of entering math environment}
	Usually, we can use \alert{\$\$}\dots\alert{\$\$} to display a maths equation instead of \samplecommand{begin}\{equation*\}\dots\samplecommand{end}\{equation*\}, which almost have same effect.\\[0.5em]
	However, there is another style of math environment, inline style, which will display the maths equation on the same line of the text before it. It is used like \alert{\$}\dots\alert{\$}
	\begin{example}
		This is a simple equation $$x^2+y^2=1$$
		This is a simple inline equation $x^2+y^2=1$	 \\
		The concentration of [H$_3$O$^+$]
	\end{example}
	This is a simple equation \alert{\$\$}x\^{}2+y\^{}2=1\alert{\$\$}\\
	This is a simple inline equation \alert{\$}x\^{}2+y\^{}2=1\alert{\$}\ \textbackslash\textbackslash\\
	The concentration of [H\alert{\$}\_3\$O\$\^{}+\alert{\$}]
\end{frame}

\begin{frame}
	\frametitle{The difference between inline and normal}
	Actually, the display style of inline and normal equations have some differences.
	\begin{example}
		\begin{minipage}{0.48\linewidth}
			\centering Expression
		\end{minipage}
		\begin{minipage}{0.24\linewidth}
			\centering inline
		\end{minipage}
		\begin{minipage}{0.24\linewidth}
			\centering normal
		\end{minipage}
		\vfill
		\begin{minipage}{0.48\linewidth}
			\samplecommand{left(}\samplecommand{frac}\{1\}\{\samplecommand{frac}\{1\}\{2\}\}\samplecommand{right)}\\
			\^{}\{\samplecommand{frac}\{1\}\{2\}\}\}
		\end{minipage}
		\begin{minipage}{0.24\linewidth}
			\centering $\left(\frac{1}{\frac{1}{2}}\right)^{\frac{1}{2}}$
		\end{minipage}
		\begin{minipage}{0.24\linewidth}
			$$\left(\frac{1}{\frac{1}{2}}\right)^{\frac{1}{2}}$$
		\end{minipage}
		\vfill
		\begin{minipage}{0.48\linewidth}
			\samplecommand{lim}\_\{n\samplecommand{to}\samplecommand{infty}\}a\_n=+\samplecommand{infty}
		\end{minipage}
		\begin{minipage}{0.24\linewidth}
			\centering $\lim_{n\to\infty}a_n=+\infty$
		\end{minipage}
		\begin{minipage}{0.24\linewidth}
			$$\lim_{n\to\infty}a_n=+\infty$$
		\end{minipage}
		\vfill
		\begin{minipage}{0.48\linewidth}
			\samplecommand{sum}\_\{k=1\}\^{}\{10\}k=55
		\end{minipage}
		\begin{minipage}{0.24\linewidth}
			\centering $\sum_{k=1}^{10}k=55$
		\end{minipage}
		\begin{minipage}{0.24\linewidth}
			$$\sum_{k=1}^{10}k=55$$
		\end{minipage}
	\end{example}
\end{frame} 

\begin{frame}
	However, most of the differences can be fixed by some other commands
	\begin{example}
		\begin{minipage}{0.48\linewidth}
			\centering Expression
		\end{minipage}
		\begin{minipage}{0.24\linewidth}
			\centering inline
		\end{minipage}
		\begin{minipage}{0.24\linewidth}
			\centering normal
		\end{minipage}
		\vfill
		\begin{minipage}{0.48\linewidth}
			\samplecommand{left(}\samplecommand{dfrac}\{1\}\{\samplecommand{frac}\{1\}\{2\}\}\samplecommand{right)}\\
			\^{}\{\samplecommand{frac}\{1\}\{2\}\}\}
		\end{minipage}
		\begin{minipage}{0.24\linewidth}
			\centering $\left(\dfrac{1}{\frac{1}{2}}\right)^{\frac{1}{2}}$
		\end{minipage}
		\begin{minipage}{0.24\linewidth}
			$$\left(\dfrac{1}{\frac{1}{2}}\right)^{\frac{1}{2}}$$
		\end{minipage}
		\vfill
		\begin{minipage}{0.48\linewidth}
			\samplecommand{lim}\samplecommand{limits}\_\{n\samplecommand{to}\samplecommand{infty}\}a\_n=+\samplecommand{infty}
		\end{minipage}
		\begin{minipage}{0.24\linewidth}
			\centering $\lim\limits_{n\to\infty}a_n=+\infty$
		\end{minipage}
		\begin{minipage}{0.24\linewidth}
			$$\lim\limits_{n\to\infty}a_n=+\infty$$
		\end{minipage}
		\vfill
		\begin{minipage}{0.48\linewidth}
			\samplecommand{sum}\samplecommand{limits}\_\{k=1\}\^{}\{10\}k=55
		\end{minipage}
		\begin{minipage}{0.24\linewidth}
			\centering $\sum\limits_{k=1}^{10}k=55$
		\end{minipage}
		\begin{minipage}{0.24\linewidth}
			$$\sum\limits_{k=1}^{10}k=55$$
		\end{minipage}
	\end{example}
	Here the command \samplecommand{limits} can be used in much more situations to fix the position of the bounds. The command \samplecommand{dfrac} is used to print a fraction in normal size.
\end{frame}

\subsection{Basic Maths Commands}

\begin{frame}
	\frametitle{Basic Maths Commands}
	Here some basic commands commonly used in \LaTeX\ are introduced. You may need \structure{amsmath} and \structure{amssymb}  packages.
	\begin{itemize}
		\item \structure{x\^{}abc}, \structure{x\_abc}, \structure{x\^{}abc\_abc} - $x^abc$, $x_abc$, $x^abc_abc$
		\item \structure{x\^{}\{abc\}}, \structure{x\_\{abc\}}, \structure{x\^{}\{abc\}\_\{abc\}} - $x^{abc}$, $x_{abc}$, $x^{abc}_{abc}$
		\item \samplecommand{sqrt}\structure{\{a\}}, \samplecommand{sqrt}\structure{[b]\{a\}} - $\sqrt{a}$, $\sqrt[b]{a}$
		\item \samplecommand{overline}\structure{\{a+b\}}, \samplecommand{underline}\structure{\{a+b\}} - $\overline{a+b}$, $\underline{a+b}$
		\item \samplecommand{overbrace}\structure{\{1+2+\samplecommand{cdots}+n\}\^{}n} - $\overbrace{1+2+\cdots+n}^n$
		\item \samplecommand{underbrace}\structure{\{1+2+\samplecommand{cdots}+n\}\_n} - $\underbrace{1+2+\cdots+n}_n$
		\item \samplecommand{overrightarrow}\structure{\{a+b\}}, \samplecommand{vec}\structure{\{a+b\}} - $\overrightarrow{a+b}$, $\vec{a+b}$
		\item \samplecommand{dots}, \samplecommand{cdot}, \samplecommand{cdots}, \samplecommand{vdots}, \samplecommand{ddots} - $\dots$, $\cdot$, $\cdots$, $\vdots$, $\ddots$
	\end{itemize}
\end{frame}

\begin{frame}
	\begin{itemize}
		\item \samplecommand{sum}\structure{\_\{k=1\}\^{}\{10\}a\_k} - $\sum\limits_{k=1}^{10}a_k$
		\item \samplecommand{prod}\structure{\_\{k=1\}\^{}\{10\}a\_k} - $\prod\limits_{k=1}^{10}a_k$
		\item \samplecommand{int}\structure{\_a\^{}bx\^{}2dx}, \samplecommand{oint} ,\samplecommand{iint}\samplecommand{limits}\structure{\_A\^{}\{\samplecommand{quad} B\}} - $$\int_a^bx^2dx\qquad\oint\qquad\iint\limits_A^{\quad B}$$
		\item \samplecommand{lim}\structure{\_\{n\samplecommand{to}\samplecommand{infty}\}\samplecommand{sup} a\_n} - $\lim\limits_{n\to\infty}\sup a_n$
		\item \samplecommand{cos}, \samplecommand{sin}, \samplecommand{tan}, \samplecommand{arccos}, \samplecommand{arcsin}, \samplecommand{arctan}, \samplecommand{log}, \samplecommand{ln} - $\cos$, $\sin$, $\tan$, $\arccos$, $\arcsin$, $\arctan$, $\log$, $\ln$
		\item \samplecommand{rightarrow}, \samplecommand{leftarrow}, \samplecommand{leftrightarrow}, \samplecommand{Rightarrow}, \samplecommand{longleftarrow}, \samplecommand{Longleftrightarrow} - $\rightarrow$, $\leftarrow$, $\leftrightarrow$, $\Rightarrow$, $\longleftarrow$, $\Longleftrightarrow$
	\end{itemize}
\end{frame}

\subsection{Matrix}

\begin{frame}
	\frametitle{Matrix environment}
	Matrix is commonly used in Maths. There are various kinds of matrix environments defined in \structure{amsmath} package, they are \structure{matrix}, \structure{pmatrix}, \structure{bmatrix}, \structure{Bmatrix}, \structure{vmatrix}, \structure{Vmatrix}.
	\begin{command}
		\samplebegin{[p/b/B/v/V]matrix}\\
		\qquad \begin{tabular}{cccccc}
			$\cdots$ & \structure{\&} & $\cdots$ & \structure{\&} & $\cdots$ & \samplecommand{\textbackslash} \\
			$\vdots$ & \structure{\&} & $\ddots$ & \structure{\&} & $\vdots$ & \samplecommand{\textbackslash} \\
			$\cdots$ & \structure{\&} & $\cdots$ & \structure{\&} & $\cdots$ & \samplecommand{\textbackslash} \\
		\end{tabular}\\
		\sampleend{[p/b/B/v/V]matrix}\\
	\end{command}	
\end{frame}

\begin{frame}
	Here is some examples of the style of these matrix.
	\begin{example}
		\begin{minipage}{0.3\linewidth}
			\centering \structure{matrix}
			$$\begin{matrix}a&b\\c&d\\\end{matrix}$$
		\end{minipage}
		\hfill
		\begin{minipage}{0.3\linewidth}
			\centering \structure{bmatrix}
			$$\begin{bmatrix}a&b\\c&d\\\end{bmatrix}$$
		\end{minipage}
		\hfill
		\begin{minipage}{0.3\linewidth}
			\centering \structure{vmatrix}
			$$\begin{vmatrix}a&b\\c&d\\\end{vmatrix}$$
		\end{minipage}
		\vfill
		\ \\[1em]
		\begin{minipage}{0.3\linewidth}
			\centering \structure{pmatrix}
			$$\begin{pmatrix}a&b\\c&d\\\end{pmatrix}$$
		\end{minipage}
		\hfill
		\begin{minipage}{0.3\linewidth}
			\centering \structure{Bmatrix}
			$$\begin{Bmatrix}a&b\\c&d\\\end{Bmatrix}$$
		\end{minipage}
		\hfill
		\begin{minipage}{0.3\linewidth}
			\centering \structure{Vmatrix}
			$$\begin{Vmatrix}a&b\\c&d\\\end{Vmatrix}$$
		\end{minipage}
	\end{example}
\end{frame}
