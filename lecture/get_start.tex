\section{Getting Started}
\begin{frame}
	\tableofcontents[currentsection,hideothersubsections]
\end{frame}

\subsection{What is \LaTeX}

\begin{frame}
	\frametitle{What is \LaTeX}
	\begin{block}{From Wikipedia, the free encyclopedia}
		LaTeX (lah-tekh, lah-tek or lay-tek, a shortening of Lamport TeX) is a document preparation system. When writing, the writer uses plain text in markup tagging conventions to define the general structure of a document (such as article, book, and letter), to stylise text throughout a document (such as bold and italic), and to add citations and cross-references. A TeX distribution such as TeX Live or MikTeX is used to produce an output file (such as PDF or DVI) suitable for printing or digital distribution. Within the typesetting system, its name is stylised as \LaTeX.
	\end{block}
\end{frame}

\subsection{Installation of \LaTeX}
\begin{frame}
	\frametitle{Installation of \LaTeX}
	\begin{block}{Windows}
		Download TeXLive on the follwing website\\
		\href{http://mirror.hust.edu.cn/CTAN/systems/texlive/Images/}{\color{blue}\underline{http://mirror.hust.edu.cn/CTAN/systems/texlive/Images/}}
	\end{block}
	\begin{block}{Linux}
		For example, on Ubuntu (or Debian), Enter the command\\
		\alert{sudo apt-get install texlive-full}
	\end{block}
	\begin{block}{MacOS}
		Download MacTeX on the following website\\
		\href{http://tug.org/mactex/mactex-download.html}
		{\color{blue}\underline{http://tug.org/mactex/mactex-download.html}}
	\end{block}
\end{frame}

\subsection{Selection of IDEs}

\begin{frame}
	\frametitle{Selection of IDEs}
	There are various IDEs recommended that support \LaTeX , for example\\
	\begin{block}{Texmaker}
		\href{http://www.xm1math.net/texmaker/}{\color{blue}\underline{http://www.xm1math.net/texmaker/}}
	\end{block}
	\begin{block}{Sublime Text}
		\href{http://www.sublimetext.com/}{\color{blue}\underline{http://www.sublimetext.com/}}
	\end{block}
	\begin{block}{Tex Studio}
		\href{http://www.texstudio.org/}{\color{blue}\underline{http://www.texstudio.org/}}
	\end{block}
	They all have cross-platform support for Windows, Linux and MacOS.
\end{frame}

\subsection{Documentation}

\begin{frame}
	\frametitle{Documentation on your computer}
	If you've installed a full version of TeXLive (as strongly recommended), the \LaTeX\ documentation about all you want to is in front of you.\\
	\ \\
	Open the command line and input the command\\
	\alert{texdoc} \structure{docname}\\
	For example, you can use the following types for the \structure{docname}\\
	\begin{description}
		\item[tex] 		A documentation about \structure{TeX}\\
		\item[article] 	A documentation about documentclass \structure{article}\\
		\item[beamer] 	A documentation about documentclass \structure{beamer}\\
		\item[pgf]		A documentation about \structure{TikZ} and \structure{PGF} (used to draw graphs)\\
	\end{description}
	Just try to \alert{texdoc} about all new things then you will be an expert in \LaTeX.
\end{frame}